%
% Copyright (c) 2014, Jonathan Anderson <jonathan.anderson@mun.ca>.
% All rights reserved.
%
% Redistribution and use in source and binary forms, with or without
% modification, are permitted provided that the following conditions are met:
%
% 1. Redistributions of source code must retain the above copyright notice,
% this list of conditions and the following disclaimer.
%
% 2. Redistributions in binary form must reproduce the above copyright notice,
% this list of conditions and the following disclaimer in the documentation
% and/or other materials provided with the distribution.
%
% THIS SOFTWARE IS PROVIDED BY THE COPYRIGHT HOLDERS AND CONTRIBUTORS "AS IS"
% AND ANY EXPRESS OR IMPLIED WARRANTIES, INCLUDING, BUT NOT LIMITED TO, THE
% IMPLIED WARRANTIES OF MERCHANTABILITY AND FITNESS FOR A PARTICULAR PURPOSE
% ARE DISCLAIMED. IN NO EVENT SHALL THE COPYRIGHT HOLDER OR CONTRIBUTORS BE
% LIABLE FOR ANY DIRECT, INDIRECT, INCIDENTAL, SPECIAL, EXEMPLARY, OR
% CONSEQUENTIAL DAMAGES (INCLUDING, BUT NOT LIMITED TO, PROCUREMENT OF
% SUBSTITUTE GOODS OR SERVICES; LOSS OF USE, DATA, OR PROFITS; OR BUSINESS
% INTERRUPTION) HOWEVER CAUSED AND ON ANY THEORY OF LIABILITY, WHETHER IN
% CONTRACT, STRICT LIABILITY, OR TORT (INCLUDING NEGLIGENCE OR OTHERWISE)
% ARISING IN ANY WAY OUT OF THE USE OF THIS SOFTWARE, EVEN IF ADVISED OF THE
% POSSIBILITY OF SUCH DAMAGE.
%

%
% The courseoutline class currently accepts two options:
%
%   - engineering:  Use the names "Engineering", "Faculty of Engineering", etc.
%                   in the relevant places and emit the Faculty's policy on
%                   academic integrity ("Like Professional Engineers...").
%   - safelabs:     Don't emit a "Lab Safety" section even if a lab slot is
%                   declared with \lab{}{}. This is intended for use with
%                   computer labs (not electrical, structural, chemical labs).
%
\documentclass[engineering]{courseoutline}

\coursename{Fundamentals of Widgets}
\coursenumber{1000}
\courseterm{Fall 2014}
\prerequisites{MATH1000}

\officialdescription{
Engineering 1000 Fundamentals of Widgets provides first-year students
with introductory exposure to the analysis and design of widgets.
Topics include widget geometry, statics, dynamics and the application
of electricity to non-conductive widgets.
}
%\fulldescription{a longer, perhaps less official, description}

\website{http://www.engr.mun.ca/teaching/ENGI1000}
%\communication{Distance learning / web submission details}

\instructor{Your name}
\email{your.email@provider.com}
\phone{864-1234}
\office{EN1234}
\officehours{Tue 15:00--16:00}

\ta{Ingrid M Engineer}
\tamail{imeng@mun.ca}
\taphone{(123) 456-7890 x9876}
\taoffice{EN1235}
\tahours{Thu 15:00--16:00}

\lecture{MWF X:00--Y:00}{EN-2001}
\tutorial{Tue 14:00--15:00}{EN-2002}
\lab{Fri 14:00--17:00}{EN-2003}

\begin{document}

\maketitle

\section{Major Topics}

\begin{itemize}
\item Widget fundamentals
\item Widget analysis
\item Widgets and society
\item Advanced widget design
\end{itemize}


\section{Learning Outcomes}

Upon successful completion of this course, the student will be able to:

\begin{enumerate}
\item explain the theoretical foundations of widgets,
\item analyze the design and application of a standard widget,
\item design simple widgets,
\item synthesize widget specifications and
\item evaluate the fitness of a widget for a specification.
\end{enumerate}


\section{Assessment}

Assignments will be given \dots
Written evaluation will take place in October and during the final exam
weeks (8--17~Dec).
The month of November should largely be used for project work.

\vspace{1em}
\begin{tabular}{llr}
  \head{Assignments (5/6)} &              & 10\% \\
  \qquad Assignment 0      & 15 September &      \\
  \qquad Assignment 1      & 22 September &      \\
  \qquad Assignment 2      & 29 September &      \\
  \qquad Assignment 3      &  6 October   &      \\

\head{Quiz}                &  3 October   &  5\% \\
\head{Midterm exam}        & 17 October   & 15\% \\
\head{Project}             &  1 December  & 20\% \\
\head{Final Exam}          &              & 50\%
\end{tabular}


\boilerplate

\end{document}
