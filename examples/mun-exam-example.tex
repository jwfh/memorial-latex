\documentclass[ece,answers,bw,legalpaper]{mun-exam}

\coursenumber{1000}
\coursename{Exam Writing}
\examdate{41 Jan 2090}
\instructions{
\begin{enumerate}
  \item This is a closed-book exam: written aids are not permitted.
  \item Calculators, phones and all other electronic aids are forbidden.
\end{enumerate}
}

\lstset{
	aboveskip=1.5em,
	belowskip=1.5em,
	frame = single,
	basicstyle = \ttfamily,
	showspaces = false,
	showtabs = false,
	showstringspaces = false,
}

\ifblackandwhite
\lstset{
	keywordstyle = \bf \ttfamily,
}
\else
\lstset{
	backgroundcolor = \color{Ivory},
	keywordstyle = \color{OliveGreen},
	commentstyle = \color{blue},
	stringstyle = \color{Purple},
}
\fi

\lstdefinestyle{console}{numbers = none}%, backgroundcolor = \color{Gainsboro}}

\ifblackandwhite
\else
\lstdefinestyle{console}{numbers = none, backgroundcolor = \color{Gainsboro}}
\fi

\lstdefinelanguage{shell}{
}

\newcommand{\cpp}[1]{\lstinline[language=C++]{#1}}
\newcommand{\file}[1]{\lstinline{#1}}
\newcommand{\shell}[1]{\lstinline[language=shell]{#1}}



\begin{document}

\maketitle

\begin{questions}

\fullwidth{\section*{In which questions varied and sundry are asked}}

\question Explain the differences between: \droptotalpoints

\begin{parts}

  \part[2] Pointers vs references
  \begin{solutionordottedlines}[4em]
    A pointer is a number, a location in memory, so it must be
    \textit{dereferenced} to access a value.
    A reference can be treated just like the thing it references.
  \end{solutionordottedlines}

  \part[2] \cpp{static_cast} vs \cpp{dynamic_cast}
  \begin{solutionordottedlines}[4em]
    The compiler has all information required to do a \cpp{static_cast}
    at compiler time, whereas a \cpp{dynamic_cast} requires run-time information.
  \end{solutionordottedlines}

\end{parts}


\vspace{1em}
\question Define the following terms. \droptotalpoints

\begin{parts}

  \part[2] Expression
  \begin{solutionordottedlines}[4em]
    Something that can be evaluated
  \end{solutionordottedlines}

  \part[2] Abstract class
  \begin{solutionordottedlines}[4em]
    A class with at least one unimplemented method
  \end{solutionordottedlines}

\end{parts}


\fullwidth{\section*{In which the student is subjected to further examination}}

\question What is the correct C++ syntax for: \droptotalpoints

\begin{parts}

  \part[1] Declaring a move assignment operator within class \cpp{Foo}?
  \begin{checkboxes}
    \choice \cpp{operator = (Foo&);}
    \choice \cpp{operator = (const Foo&);}
    \CorrectChoice \cpp{operator = (Foo&&);}
    \choice \cpp{operator = (const Foo&&);}
  \end{checkboxes}

  \part[1] Creating a subclass of \cpp{class Parent}?
  \begin{checkboxes}
    \CorrectChoice \lstinline[language=C++]|class Child : public Parent {};|
    \choice \lstinline[language=C++]|class Child : override Parent {};|
    \choice \lstinline[language=C++]|class Child : subclass Parent {};|
    \choice \lstinline[language=C++]|class Child : virtual Parent {};|
  \end{checkboxes}

  \part[1] Defining a lambda function that increments an integer parameter?
  \begin{checkboxes}
    \choice \lstinline[language=C++]|[]() { return x + 1; }|
    \choice \lstinline[language=C++]|[x]() { return x + 1; }|
    \choice \lstinline[language=C++]|[x](int x) { return x + 1; }|
    \CorrectChoice \lstinline[language=C++]|[](int x) { return x + 1; }|
  \end{checkboxes}

\end{parts}

\vspace{1em}   % TODO: increase pre-question spacing automatically

\question What is invalid about each of the following?
          How could you fix it? \droptotalpoints

\begin{parts}
  \part[2]
    Casting a \cpp{Car*} to a \cpp{Subaru*}:

    \begin{lstlisting}[language=C++]
class Car { /* ... */ };
class Subaru : public Car { /* ... */ };
Car *car = /* ... */;
Subaru *s = car;
    \end{lstlisting}
    \begin{solutionordottedlines}[4em]
      The \cpp{Car} might not be a \cpp{Subaru}.
      Use \cpp{dynamic_cast} to check.
    \end{solutionordottedlines}

\end{parts}


\pagebreak


\fullwidth{\section*{Bonus questions}}

\bonusquestion

\begin{parts}
  \bonuspart[1]
    What is the simplest possible lambda function?

  \begin{solutionordottedlines}[2em]
    \lstinline[language=C++]|[](){}|
  \end{solutionordottedlines}

\end{parts}

\end{questions}


\end{document}

